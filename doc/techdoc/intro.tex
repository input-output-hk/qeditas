This document is intended as a reference
for those wanting to understand, modify or extend the
code supporting Qeditas~\cite{White2015c}.
The document is currently under construction (in November 2015),
as is the Qeditas code itself.

Qeditas is intended to be a realization of the QED Project~\cite{QED}
to construct a library of formalized mathematics.
For those wanting to learn more about formalized
mathematics~\cite{Barendregt2005} is a good starting point.
Some popular systems for formalizing mathematics include
Mizar~\cite{RudnickiTrybulec2001},
Isabelle~\cite{Nipkow-Paulson-Wenzel:2002},
various HOL systems~\cite{Gordon91,harrison-hollight},
Coq~\cite{Coq:manual} and
Agda~\cite{Norell08}.
Some large formalizations of major mathematical results
are described in~\cite{Gonthier2007}, \cite{Gonthier2013}
and
\cite{Flyspeck2015}.

Qeditas uses a block chain and distributed consensus system to maintain both a library
of formal publications
and information about who made a definition or proved a theorem.
In addition, the block chain maintains and enforces a rights management
system determining conditions under which an object can be imported and used
(without repeating the definition)
and when a theorem can imported as something known (without needing to reprove it).
The first network with a block chain and distributed consensus system
was Bitcoin and Nakamoto's white paper~\cite{Nakamoto2008} provides a good introduction.
Qeditas uses a proof of stake consensus system~\cite{ProofOfStakeDefinite}
combined with proof of storage (similar to proof of retrievability~\cite{MillerJSPK14}).
Proof of stake is sensitive to the initial distribution, and Qeditas
has an initial distribution based on a snapshot of the Bitcoin block chain.
This makes Qeditas a Bitcoin ``spin-off''~\cite{Spinoff,PreSpinoff}.

\section{Code Repository}

The main git repository is publicly available:
\begin{verbatim}
git clone git://qeditas.org/qeditas.git
\end{verbatim}

There are three primary git branches: \master, {\dev} and \testing.
The {\master} branch is intended to contain the code which can be tagged to create
specific Qeditas versions.
The {\dev} branch is where code can be written, modified and to some degree tested.
The {\testing} branch contains a number of unit tests.
The code in {\dev} should regularly be merged into {\testing} and
the unit tests run to ensure the unit tests still pass.
In addition, new unit tests should be added as new code is added to the {\dev} branch.
Unfortunately, the {\dev} branch has developed significantly since it was last merged into the {\testing} branch,
and so the code and unit tests in the
{\branch{testing}} branch is out of date with respect to the code in the {\branch{dev}} and {\branch{master}} branches.

When the code in {\dev} is stable, it should be merged to \master.

The source for this document is also part of the {\dev} branch and
the intention is for it to correspond to the code in the {\dev} branch.
Likewise, the source for this document can be merged into the {\master} branch
when appropriate.

Another branch, \initdistr, contains code for computing the ledger tree for
the initial distribution of Qeditas currency units. This distribution
was based on a snapshot of the Bitcoin block chain.



\section{Qeditas Theory in Coq}

There is also a separate formal Qeditas related development in Coq
in which certain properties were proven.
The git repository is named {\tt{qeditastheory}}
and is also publicly available:
\begin{verbatim}
git clone git://qeditas.org/qeditastheory.git
\end{verbatim}
The Coq code is somewhat out of date since some aspects of Qeditas have changed
in the meantime. Nevertheless the Coq development should usually indicate how
different data types were intended to be used and what properties certain functions
were meant to have. In appropriate places we will point not only to the Qeditas
OCaml code, but also corresponding code in the Coq version.

\section{License and Credit}

Qeditas is an open source project and all code and documentation is released
under the MIT License. The code and documentation is attributed to
``The Qeditas developers'' rather than giving a list of names or handles.
Some open source developers advise not to include names and handles inside
code comments. Having a name in the code might suggest a kind of ``ownership''
that may make others uncomfortable making modifications to the code.
If individual coders wish to record their contributions to the
code, the appropriate place would be in this technical documentation.
(For example, most of the initial code for Qeditas and the
first version of this documentation was written
by Bill White in 2015 with significant additions and modifications by Trent Russell in 2016.)
Also, if code is taken or ported from other open source projects, this
should be noted in this document.
(For example, the code for elliptic curves and proof checking was taken from Egal~\cite{Brown2014}
and this is noted in the appropriate sections.)
