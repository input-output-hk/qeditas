The module \module{utils} (\file{utils.ml} and \file{utils.mli}) defines a few simple functions used by many other modules.
The current code is for handling the log file, for computing the ``era'' and corresponding maximum block size,
and for initializing and using OCaml's Random module. More details about each of these follow.

\section{Logging}

Qeditas logs information to a {\file{log}} file which should be in the main directory given by {\var{datadir}} (presumably {\dir{.qeditas}}) or the {\dir{testnet}} subdirectory.
The function {\func{openlog}} and {\func{closelog}} open and close the log file.
The relevant out\_channel is the value held in the {\val{log}} ref cell.

(Note: At the moment Qeditas logs an excessive amount of information intended for debugging purposes.)

\section{Eras and Block Size}

The block height determines what ``era'' Qeditas is in, as computed by the {\func{era}} function.
The era is used to determine the block rewards and the maximum block delta size. (A ``block'' consists of a ``header'' and a ``delta.'')
The initial era is ``Era 1'' which lasts from blocks 1 through 70000 (when the reward is 25 fraenks
and the maximum block delta size is 500,000 bytes).
The next 41 eras (``Era 2'' through ``Era 42'') each last 210000 blocks, at which time the reward halves
and the maximum block delta size doubles.
Starting at block 8680001 (in roughly 165 years after block one) the final era (``Era 43'') begins.
In this final era the block reward is 0 and the maximum block delta size is 512 megabytes.

The code for computing the maximum block delta size ({\func{maxblockdeltasize}}) is included in the \module{utils} module
because it is needed in the \module{net} module to determine the maximum message size.
On the other hand, the function for computing the reward ({\func{rewfn}}) is not given until the module \module{block}.

\section{Randomness}

The boolean {\var{random\_initialized}} is set to {\val{false}} until {\func{initialize\_random\_seed}} has been called.
{\func{initialize\_random\_seed}} is called the first time a random value is requested.\footnote{This way of doing things was so that, in principle, a user could use Qeditas in a way that does not require randomness (e.g., never signing a block or transaction and never generating a private key). In practice a random number is requried by {\func{initialize}} in {\file{qeditas.ml}} to set the node's nonce -- {\var{this\_nodes\_nonce}} in {\module{net}}.}

The function {\func{initialize\_random\_seed}} calls {\func{Random.full\_init}} using some data (or raises an exception).
The data is either the value of {\var{randomseed}} or 32 bytes from {\file{/dev/random}}.
If {\var{randomseed}} is not set and {\file{/dev/random}} does not exist, then a Failure exception is raised.

The functions {\func{rand\_bit}}, {\func{rand\_int32}} and {\func{rand\_int64}}
use OCaml's Random module to obtain a boolean, int32 or int64, respectively.
A corresponding function {\func{rand\_256}} in the module \module{sha256} generates a 256-bit random number
using OCaml's Random module. (It is in this later module because it uses code to create a \type{big\_int}
from 8 int32 values.) There is also a (currently unused) function {\func{strong\_rand\_256}} in the module \module{sha256}
which obtains the 256-bit random \type{big\_int} using data from {\file{/dev/random}}.
