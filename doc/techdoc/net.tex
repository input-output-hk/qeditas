{\bf{Warning:}} The networking code is only partially written and may possibly be rewritten completely.
The information is this chapter is subject to changes in the near future.

The intention of the {\module{net}} module is to create and handle connections
to remote nodes.
We briefly describe what exists, but the code is unusable in its current form.
The networking code should be rethought, redesigned and rewritten from scratch.

\section{State}

There are a number of variables to keep up with the state of the system.
It is not clear these variables belong in this module.
\begin{itemize}
\item {\var{recentblockheaders}} is a list of recent block headers sorted according to cumulative stake.
This ensures the head of the list is the recent block header with the most cumulative stake.
\item {\var{recentledgerroots}} associates ledger roots (hash values) with abbreviations giving where the corresponding {\type{ctre}} has been saved. This is needed to restore the ledger tree into memory when it is needed.
\item {\var{recentcommitments}} is intended to store what commitments have been made by staking blocks to avoid double signing (signing blocks on two branches stemming from a recent fork).
\item {\var{recentblockdeltahs}} lists a skeleton form of recent block deltas, where the transactions in the block delta are refered to by their hash.
\item {\var{recentblockdeltas}} lists recent block deltas.
\item {\var{recentstxs}} lists recent transactions which may be included in a block.
\item {\var{waitingblock}} lists block headers which are not considered ``valid,'' but may be valid later (e.g., the time stamp is in the future).
\item {\var{preconns}} lists connections in the handshake phase before other messages can be exchanged.
\item {\var{conns}} lists active connections.
\item {\var{this\_nodes\_nonce}} is a random nonce to prevent nodes from connecting to themselves.
\end{itemize}

The function {\func{insertnewblockheader}} inserts a new block header into the sorted
list {\var{recentblockheaders}}.
This makes the ``valid'' block header with the most cumulative stake at the head of the
list. The intention is that blocks should be staked on top of this ``best'' block.

\section{Making Connections}

The function {\func{myaddr}} returns the address to optionally listen for connections on.
This must be specified in the configuration or the node will not listen for connections.

The function {\func{openlistener}} starts a listener socket to wait for incoming connections.

The function {\func{accept\_nohang}} checks if a node is attempting to make a connection
without waiting for a connection (i.e., with a timeout).

The exception {\exc{EnoughConnections}} is raised if a connection is being rejected
because there are enough connections already.
The exception {\exc{RequestRejected}} may be raised if the connection fails, e.g.,
if the SOCKS4 protocol indicates a rejection.

The functions {\func{connectpeer}} and {\func{connectpeer\_socks4}}
try to initiate a connection to a listening peer.
The function {\func{tryconnectpeer}} searches for a known peer to try to connect to.
The function {\func{getfallbacknodes}} lists fallback nodes to try if no other connection works.

The function {\func{initialize\_conn\_accept}} is called after the listener finds
a node trying to connect.
Both put the initial connection onto {\var{preconns}}
and attempt to perform the initial handshake. Only after the initial handshake
succeeds does the connection become a legitimate connection listed
in {\var{conns}}.

The functions
{\func{addknownpeer}},
{\func{getknownpeers}},
{\func{loadknownpeers}} and
{\func{saveknownpeers}}
are to keep up with information about peers in the recent past.

The function {\func{extract\_ip\_and\_port}} extracts an ip-address and a port number
from a string, with a boolean to indicate if the address is ipv6.

The function {\func{sethungsignalhandler}} sets the timeout handler to raise the exception {\exc{Hung}}
upon timeout. It is not clear why this should be exported.

\section{Messages and Communication}

Messages are elements of type {\type{msg}}.
The kinds of messages largely follow Bitcoin's messages, except there
are additional messages:
\begin{itemize}
\item ${\constr{GetFramedCTree}}(v,b,t,f)$
gives a block height $b$, a compact tree $t$ (of type {\type{ctre}})
and a remote frame $f$ (of type {\type{rframe}}).
The intention is to claim that $t$ is the approximation of the compact tree at height $b$
where $f$ outlines the approximation.
The 32-bit integer $v$ is a ``version number'' which is only included in case it's needed later.
(Bitcoin messages often have such a version number.)
\item ${\constr{MCTree}}(v,t)$ is the compact tree $t$ (of type {\type{ctre}}).
The 32-bit integer $v$ is a ``version number'' which is only included in case it's needed later.
\item ${\constr{Checkpoint}}(b,h)$ a ``checkpoint'' that the block at height $b$ has hash $h$.
\item ${\constr{AntiCheckpoint}}(b,h)$ an ``anticheckpoint'' that the block at height $b$ does not have hash $h$.
\end{itemize}

The exception {\exc{IllformedMsg}} is raised if something is wrong with the format of a message.

The function {\func{send\_msg}} sends a message through a channel to another node.
The function {\func{send\_reply}} sends a message through a channel to another node
and marks it as a reply to a previous message.

The function {\func{input\_byte\_nohang}} listens for a byte with a timeout (returning
{\val{None}} if no byte is available to be read).
The function {\func{rec\_msg\_nohang}} tried to receive a message with two timeouts.
The first timeout indicates how long to wait before a message begins
and the second timeout indicates how long to try to read a message after
it begins but until it ends.

The recursive {\type{pendingcallback}} type gives a way of specifying
what should happen if a reply is received.

The funciton {\func{broadcast\_inv}} broadcasts an inventory message to every connection.

The function {\func{handle\_msg}} is intended to handle messages received,
but most cases are unwritten.

The function {\func{try\_requests}} searches through the current connections
to try to find one from which it can request some data.

The functions {\func{handle\_orphans}} and {\func{handle\_delayed}}
are intended to look back at orphan block headers and block headers which
were delayed for some reason (e.g., if the time stamp was too far in the future).
These were written because in testing sometimes a branch with the most cumulative stake
would not be communicated because an earlier block header had been delayed or rejected.

% {\type{msg}}

% internal serialization and deserialization functions {\serfunc{seo\_msg}} and {\serfunc{sei\_msg}}
The function {\func{send\_initial\_inv}} is called when a connection is established (after the handshake)
to send the inventory.

\section{Connection State}

The type {\type{connstate}} consists of a number of mutable fields
which represent the state of a connection.
This includes whether the node is alive,
when the last message was received,
what requests are pending,
the known inventory of the remote node
and what inventory has been sent, what inventory has been requested.
It also has the infrastructure to remember the remote frame of the other node
(indicating which parts of the ledger tree is being followed)
as well as the joins of remote frames one or two hops away. This was
intended to help find a node from which a certain part of the tree could be requested.
Other fields keep up with how much of the history of the block chain the remote node
has, and how up to date the remote node is.






