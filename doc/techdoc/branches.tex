The main git repository is publicly available:
\begin{verbatim}
git clone git://qeditas.org/qeditas.git
\end{verbatim}

There are three primary git branches: \master, {\dev} and \testing.
The {\master} branch is intended to contain the code which can be tagged to create
specific Qeditas versions.
The {\dev} branch is where code can be written, modified and to some degree tested.
The {\testing} branch contains a number of unit tests.
The code in {\dev} should regularly be merged into {\testing} and
the unit tests run to ensure the unit tests still pass.
In addition, new unit tests should be added as new code is added to the {\dev} branch.
Unfortunately, the {\dev} branch has developed significantly since it was last merged into the {\testing} branch,
and so the code and unit tests in the
{\branch{testing}} branch is out of date with respect to the code in the {\branch{dev}} and {\branch{master}} branches.

When the code in {\dev} is stable, it should be merged to \master.

The source for this document is also part of the {\dev} branch and
the intention is for it to correspond to the code in the {\dev} branch.
Likewise, the source for this document can be merged into the {\master} branch
when appropriate.

Another branch, \initdistr, contains code for computing the ledger tree for
the initial distribution of Qeditas currency units. This distribution
was based on a snapshot of the Bitcoin block chain.

