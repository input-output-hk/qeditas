The {\module{commands}} module ({\file{commands.ml}} and {\file{commands.mli}})
contains code for handling top level commands (from the command line interface).

The module {\module{commands}} is intended to support a variety of commands a user may need.
At the moment it only supports limited wallet and transaction creation commands.
Some state information is held in this module (although it likely should be moved elsewhere).
\begin{itemize}
\item {\var{walletkeys}} contains the private keys in the wallet.
More specifically it is a list of values of the form $(k,b,(x,y),w,h,a)$
where $k$ is the private key, $b$ is a boolean indicating if it is for the compressed public key,
$(x,y)$ is the public key, $w$ is the string base-58 WIF format, $h$ is the 20-byte hash value
corresponding to the p2pkh address and $a$ is the string base-58 Qeditas p2pkh address.
\item {\var{walletp2shs}} contains entries of the form $(h,a,\overline{b})$
where $h$ is the 20-byte hash value of a p2sh address, $a$ is the base-58 Qeditas p2sh address
and $\overline{b}$ is the sequence of bytes giving the script corresponding to $h$.
Note that this does not directly give a way of ``signing'' for the p2sh address.
\item {\var{walletendorsements}} contains the endorsements in the wallet.
In particular it is a list of values of the form $(\alpha,\beta,(x,y),b,\sigma)$
where $\alpha$ and $\beta$ are pay addresses,\footnote{Actually, in what is implemented we assume they are both p2pkh addresses. In principle endorsements involving p2sh addresses are supported by the code in {\module{sigant}} and {\module{script}}, but support has not been implemented in {\module{commands}}.}
$(x,y)$ is the public key for $\alpha$,
$b$ is a boolean indicating if $\alpha$ is the address for the compressed public key
and $\sigma$ is a signature for a
Bitcoin message of the form ``endorse $\beta$''
(or ``testnet:endorse $\beta$'' in the testnet)
signed with the private key for $\alpha$.
The private key for $\beta$ should be in {\var{walletkeys}}
and this private key along with the endorsement means the wallet can sign for $\alpha$.\footnote{The endorsement mechanism gives Bitcoin users a way to claim their part of the initial distribution without revealing their private keys.}
\item {\var{walletwatchaddrs}}
contains addresses to ``watch.'' 
\item {\var{stakingassets}}
contains a list of assets which the node can stake. This changes as {\var{bestnode}} from {\module{blocktree}} changes.
\item {\var{storagetrmassets}}
is intended contain a list of assets at term addresses which the node can use as proof-of-storage to improve the chances of staking.
Currently it is unused.
\item {\var{storagedocassets}}
is intended to contain a list of documents at publication assets which the node can use as proof-of-storage to improve the chances of staking.
Currently it is unused.
\item {\var{txpool}} is a hash table associating hash values (transaction ids)
with signed transactions. This is loaded and saved to the file {\file{txpool}}.
The intention is that this holds transactions which have been published
but are not yet included in a block.
\end{itemize}

The following functions are for loading and saving the state in certain files.
\begin{itemize}
\item {\func{load\_txpool}} sets {\var{txpool}} by loading the contents fo {\file{txpool}}.
\item {\func{load\_wallet}} sets {\var{walletkeys}}, {\var{walletp2shs}},
{\var{walletendorsements}}
and {\var{walletwatchaddrs}}
by loading the file {\file{wallet}}.
\item {\func{save\_wallet}} saves the current wallet contents
(the values of {\var{walletkeys}}, {\var{walletp2shs}},
{\var{walletendorsements}}
and {\var{walletwatchaddrs}})
in {\file{wallet}}.
\item {\func{printassets}} prints the assets from the current ledger tree (the ledger tree with root in the current best node) at the addresses
mentioned in the wallet.
\item {\func{printassets\_in\_ledger}} prints the assets in a ledger with an explicitly given ledger root
at the addresses mentioned in the wallet.
\item {\func{printctreeinfo}} prints a summary of information about the compact tree with the given hash root.
\item {\func{printctreeelt}} prints the ctree element (complete information for exactly 9 levels) with the given hash root.
\item {\func{printhconselt}} prints the hcons element with the given hash root, meaning it prints the asset id and optionally the root for the next hcons element.
\item {\func{printasset}} prints the asset with the given asset id.
\item {\func{printtx}} prints the transaction with the given hash root.
\item {\func{btctoqedaddr}} parses a Bitcoin address (base-58 representation) and prints the corresponding Qeditas address (base-58 representation).
\item {\func{importprivkey}} imports a private key given in Qeditas WIF.
\item {\func{importbtcprivkey}} imports a private key given in Bitcoin WIF.
\item {\func{importendorsement}} imports an endorsement.
\item {\func{importwatchaddr}} imports a Qeditas address to watch.
\item {\func{importwatchbtcaddr}} imports a Bitcoin address in order to watch the corresponding Qeditas address.
\item {\func{createsplitlocktx}} creates a transaction to split an asset
into several assets with a given lock height.
\item {\func{signtx}} signs a transaction.
\item {\func{savetxtopool}} saves a transaction to the local transaction pool (without publishing it).
\item {\func{sendtx}} publishes a transaction by sending it to peers.
\end{itemize}

