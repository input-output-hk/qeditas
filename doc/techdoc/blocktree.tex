The module {\module{blocktree}} ({\file{blocktree.ml}} and {\file{blocktree.mli}})
contains code related to keep up with the current tree of blocks
and the current best block.
Other various information is also included here.

\section{Block Tree}

The main data type in the module {\module{blocktree}} is the type {\type{blocktree}}
with one constructor {\constr{BlocktreeNode}} (described below).
Each node can be thought of as between a block and possible successor blocks.
Enough information is included in the node to determine if a new block (header) is a valid successor block (header).
The root of the tree is the ``genesis node'' which contains information about
the initial ledger tree, initial stake modifier and initial target.
Qeditas has no ``genesis block'' in the traditional sense. The first block would be
a valid child of the genesis node.

Nodes to the tree are added as headers are received (or staked) even if the corresponding block delta
has not yet been received.
In order to distinguish between these cases there is a type {\type{validationstatus}}
with three constructors: {\constr{Waiting}} (meaning the block delta is being requested from peers),
{\constr{ValidBlock}} (meaning a delta has been received and checked to be part of a valid block when
combined with the header)
or {\constr{InvalidBlock}} (meaning there is no delta making the header part of a valid block, though it is not clear how this could be determined).

The single constructor {\constr{BlocktreeNode}} of the type {\type{blocktree}} takes 13 arguments:\footnote{This probably should be a record type.}
\begin{itemize}
\item an optional {\type{blocktree}} giving the parent, which is {\val{None}} for the genesis node (and possibly for some checkpoint nodes).
\item a reference to a list of hash values, giving a list of addresses which have staked in some child of this node up to six levels deep. This is to be used to identify double-staking which can be punished by proof-of-forfeiture.
\item an optional hash of the previous block header, which is {\val{None}} for the genesis node.
\item an optional root of the current theory tree (see {\type{ttree}} in Chapter~\ref{chap:math}).
\item an optional root of the current signature tree (see {\type{stree}} in Chapter~\ref{chap:math}).
\item the current ledger hash root.
\item the current target information (including the current and future stake modifier).
\item the current time stamp.
\item the current cumulative stake.
\item the current block height.
\item a reference to the current validation status (of type {\type{validationstatus}}).
\item a reference to a boolean indicating if the block above the node has been blacklisted, so children should not be considered valid.
\item a reference to a list of children, which gets updated as successor blocks are found.
\end{itemize}
There are a variety of functions for obtaining the information above from a node in the tree:
{\func{node\_recent\_stakers}},
{\func{node\_prevblockhash}},
{\func{node\_theoryroot}},
{\func{node\_signaroot}},
{\func{node\_ledgerroot}},
{\func{node\_targetinfo}},
{\func{node\_timestamp}},
{\func{node\_cumulstk}},
{\func{node\_blockheight}},
{\func{node\_validationstatus}} and
{\func{node\_children\_ref}}.

The function {\func{eq\_node}} tests if two nodes are equal by simply checking
if they have the save previous block hash.
The usual equality operation cannot be used since the target is of type {\type{big\_int}}
whose values are (often) abstract.

The variable {\var{genesisblocktreenode}} is initialized to be the node with the
genesis ledger root, genesis target information and genesis time stamp.
A child of this node corresponds to a block at height $1$.

The hash table {\var{blkheadernode}} associates hash values with their corresponding
node in the block tree.

The functions {\func{process\_new\_header}},
{\func{process\_new\_header\_a}},
{\func{process\_new\_header\_aa}},
{\func{process\_new\_header\_ab}}
and {\func{process\_new\_header\_b}}
are used to check a header is valid and, if so, create
a node for it and enter the node into the block tree.
If the previous block hash in the header does not correspond
to a node in the tree (according to {\var{blkheadernode}}),
the new header is considered an orphan and put into
the hash table {\var{orphanblkheaders}}.
The parent header of an orphan block header are requested using {\func{find\_and\_send\_requestdata}}, if possible.
When a new header is processed, it may mean previously orphaned headers are no longer orphaned,
at which point the headers should be removed from {\var{orphanblkheaders}}
and new nodes in the block tree should be added. (The handling of orphan blocks has not been tested.)

In addition, there is a list {\var{earlyblocktreenodes}} which contains headers which have
been received but whose time stamps are in the future. These are intended to be delayed until
the appropriate time arrives (although the code to handle them later does not seem to be written).

The hash table {\var{tovalidate}} is a set of hashes of headers where we are still waiting to validate the corresponding block delta.

\section{Checkpoints}

The checkpoint public key is given in this module
by {\var{checkpointspubkeyx}}
and {\var{checkpointspubkeyy}}.
If the corresponding private key is given on the command line or in the configuration file,
then {\var{checkpointsprivkeyk}} is set to the corresponding integer.
(The key is for the uncompressed address, and this fact is hard-wired into the code.)

The value {\var{lastcheckpointnode}} is set to the most recent checkpoint, which is the
genesis node by default.

The hash table {\var{checkpoints}} associates header hashes with pairs $(h,\sigma)$
where $h$ is a block height and $\sigma$ is a signature of the hash by the checkpoint key.

\section{Best Node}

The current block chain is determined by finding the ``best node''
in the block tree, i.e., the node with the highest cumulative stake.
We sometimes allow the best node to be awaiting validation and sometimes require the best node
to be fully validated.

The variable {\var{bestnode}} is set to the current best node.
This gets reset when a new block is staked, if a new header is received
or if Qeditas gives up waiting for a node to be validated.
In the last case, we may want to find the best validated node.
The value of {\var{bestnode}} should never be updated directly,
but instead should be updated by calling {\func{update\_bestnode}}.
Given a node $n$, {\func{update\_bestnode}} sets {\var{bestnode}} to $n$,
sets {\var{netblkh}} to the current block height (technically, the
height of the next block to be found)
and, if the private key for signing checkpoints is known, the block which
is now 36 blocks deep is signed as a checkpoint,
{\var{lastcheckpointnode}} is updated
and the checkpoint is broadcast (as inventory) to all peers.

The function {\func{initblocktree}} creates a genesis block tree node
and initializes the values
{\var{genesisblocktreenode}},
{\var{lastcheckpointnode}}
and {\var{bestnode}} accordingly.
Then {\func{init\_headers}} is called to traverse all known block headers from
the block header database {\module{DbBlockHeader}}
and create corresponding nodes in the block tree.

The function {\func{find\_best\_validated\_block\_from}}
finds the descendent node with validation status {\constr{ValidBlock}}
(and which is not blacklisted)
with the highest cumulative stake.
The best node and cumulative stake found so far are extra arguments to compare against.
The function {\func{find\_best\_validated\_block}} calls {\func{find\_best\_validated\_block\_from}}
starting from the current value of {\var{lastcheckpointnode}}
and calls {\func{update\_bestnode}} with the result.

The function {\func{print\_best\_node}} gives the hash of the last block header of the current best node (or endicates it is the genesis node if there is no previous header).

The function {\func{record\_recent\_staker}} climbs an indicated number of parents
from a given node and inserts a given address into the recent stakers of the node.
This information is then used by {\func{is\_recent\_staker}} to determine if
an address has staked on some descendent of a node (up to some depth).
The relevant depth is $6$, but the functions are recursive, so the depth arguments $i$ are generic.

\section{Other Local Data}

The hash table {\var{stxpool}} associates hash values with corresponding signed transactions
and {\var{published\_stx}} records which of these transactions have been published (broadcast to peers).

The hash table {\var{thytree}} associates hash values with the root of a corresponding theory tree {\type{ttree}}.
The function {\func{lookup\_thytree}} looks up the theory tree given an optional hash value, where the default value of {\val{None}} returns the empty theory tree {\val{None}}.
For the moment, Qeditas has only been tested with empty theory trees.

Similarly, there is a hash table {\var{sigtree}} which associates hash values with the root of a corresponding signature tree {\type{stree}}.
The function {\func{lookup\_sigtree}} looks up the signature tree given an optional hash value, where the default value of {\val{None}} returns the empty signature tree {\val{None}}.
Again, Qeditas has only been tested with empty signature trees.

\section{Networking Code}

Since the {\module{blocktree}} module has access to the current state of the block chain,
a function {\func{send\_inv}} is defined to collect the most recent inventory data
(transactions, block headers and block deltas) to send to new peers.
The networking code obtains access to the {\func{send\_inv}} function via the reference
{\var{send\_inv\_fn}} which is set to {\func{send\_inv}} in {\module{blocktree}}.

The functions {\func{publish\_stx}} and {\func{publish\_block}}
add transactions and blocks to the local database
and broadcast them as new inventory to peers.

In addition, most of the implemented message handlers for the networking code (see Chapter~\ref{chap:net})
are given in the {\module{blocktree}} module by associating functions with message types in
the hash table
{\var{msgtype\_handler}}.
Here are the message types handled in {\module{blocktree}}:
\begin{itemize}
\item {\constr{Inv}}: The payload of an inventory message should be a 32-bit integer $n$ indicating
how many inventory items there are followed by $n$ triples $(\tau,h,k)$
where $\tau$ is a message type (as a byte, indicating the kind of object in the inventory), $h$ is a block height (as a 64-bit integer)
and $k$ is a hash value (indicating the identifier of the object in the inventory).
The handler reads these triples and determines whether or not to request the corresponding objects.
Headers will be requested if they are new and those to be requested are collect into a list to be requested in batches of at most 255 headers (by {\func{req\_header\_batches}}).
Block deltas are requested if they are new and the corresponding header is waiting to be validated (as indicted by {\var{tovalidate}}).
If a signed transaction is new, then the corresponding unsigned transaction is requested.
If the transaction of a signed transaction is known but the signatures are not, the signatures are requested.
A checkpoint is requested unless it is a known checkpoint or would be an ancestor of the current checkpoint.
No other inventory is immediately requested.
All inventory is inserted into the {\field{rinv}} field of the connection state in case
we need to request something later.
\item {\constr{GetHeader}}: The payload contains a hash value.
If the corresponding header has already been sent to the node (as recorded in {\field{sentinv}}),
then it is not resent. Otherwise, the header is loaded from the database and sent to the peer in a {\constr{Headers}} message. If the header is not in the database, the message is ignored.
\item {\constr{GetHeaders}}: The payload contains a byte $n$ indicating the number of headers being requested, followed by $n$ hash values.
The corresponding headers which have not yet been sent are collected and send to the peer in a {\constr{Headers}} message.
\item {\constr{Headers}}: The payload contains a byte $n$ indicating the number of headers being sent, followed by $n$ hash values and block headers.
Each header must hash to the given hash value, otherwise this is logged and the connection is marked as banned by setting {\field{banned}} in the {\type{connstate}} to {\val{true}}.
Each new header is checked to be valid and, if so, processed by {\func{process\_new\_header\_ab}} (which assumes the given header is valid).
If a received header is not valid, this is logged and the connection is marked as banned by setting {\field{banned}} in the {\type{connstate}} to {\val{true}}.
\item {\constr{GetBlockdelta}}: The payload contains a hash value identifying a block.
If the corresponding block delta has not yet been sent to the peer, it is sent in a {\constr{Blockdelta}} message.
\item {\constr{Blockdelta}}: The payload contains a hash value and a block delta.
If the hash value is of a block for which the block delta is already known, then
the handler stops (before deserializing the block delta).
Otherwise, the node for the corresponding header is found in the block tree.
If there is no such node in the block tree, the message is ignored.
Assume there is such a node $n$.
We check its validation status.
The case we are interested in is when the validation status is {\constr{Waiting}}
but there is not yet a candidate {\type{blockdelta}} to validate.
In this case we find the parent node $p$ of $n$.
If the parent node is already known to correspond to a valid block (or there is no parent node),
then the block delta is deserialized
and the function {\func{validate\_block\_of\_node}} is called to check
that the new block is valid and update the validation status of the node.
(Assuming the block is valid, {\func{validate\_block\_of\_node}} also traverses children
which have block deltas waiting to be validated and tries to validate these.)
If the parent node has not yet been validated,
then the block delta is deserialized and stored (along with the connection state of the peer)
with the {\constr{Waiting}} status. In principle it will be validated after its ancestors
have been validated.
\item {\constr{GetTx}}: 
\item {\constr{Tx}}: 
\item {\constr{GetTxSignatures}}: 
\item {\constr{TxSignatures}}: 
\item {\constr{GetCheckpoint}}: 
\item {\constr{Checkpoint}}: 
\end{itemize}

\section{Dumping the State}

The function {\func{dumpblocktreestate}} writes a large amount of information about
the current block tree state into a channel (presumably of an open text file).
This is for debugging purposes.
