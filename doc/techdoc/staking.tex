The staking code is part of the top level file {\file{qeditas.ml}}.
A thread is created via the function {\func{stakingthread}}
if the configuration variable {\var{staking}} is set to {\val{true}}.
{\func{stakingthread}} is a loop which takes the current
value of {\val{bestnode}} and tries to find an asset which
can stake the next block within the next hour or two.

The hash table {\val{nextstakechances}} associates block header hash values with
information about the next chance to stake on top of the corresponding block.
The information is of type {\type{nextstakeinfo}} which has to constructors:
\begin{itemize}
\item ${\constr{NextStake}}(i,\alpha,h,b,\omega,v,c)$ meaning the next chance to stake
is at type $i$ with the currency asset worth $v$ cants with id $h$, birthday $b$ and obligation $\omega$ held at (p2pkh) address $\alpha$
and that the resulting block will have cumulative stake $c$.
\item ${\constr{NoStakeUpTo}}(i)$ meaning there is no chance to stake up to time $i$.
\end{itemize}
The entries in {\var{nextstakechances}} are computed
by the function {\func{compute\_staking\_chances}}.
{\func{compute\_staking\_chances}} takes a {\type{blocktree}} node $n$ 
and two integers $t_0$ and $t_1$ giving the starting time and ending time and searches for the first
$i\in [t_0,t_1]$ where some held currency asset can stake.
First, it collects the assets held at p2pkh addresses where either a private key
or an endorsement is in the wallet. These assets are put into {\val{stakingassets}}
from the {\module{commands}} module.
If {\func{compute\_staking\_chances}} finds an asset that can stake, the relevant information is put into {\val{nextstakechances}}.
If none is found, ${\constr{NoStakeUpTo}}(t_1)$ is put into {\val{nextstakechances}}.

Assume there is an asset which can stake.
If the next chance to stake is greater than 10 seconds away, the thread sleeps for a minute
and checks again (in case the best node has changed in the past minute).
If it is within 10 seconds of the time to stake, the block header and delta are constructed
and published.\footnote{A final validity check is done before publishing the block. In principle
all blocks constructed should be valid, but in practice sometimes invalid blocks are constructed.
This is presumably due to bugs.}

If no asset can stake within the next hour, the thread continues sleeping for 1 minute
before rechecking the current state. Eventually either the best node changes or
enough time has past that we call {\func{compute\_staking\_chances}} again for more information.

In some cases the exception {\exc{StakingProblemPause}} is raised. If this happens,
it is considered a problem caused by a bug. The staking thread deals with this by pausing
the thread for an hour and then trying to stake again in the new state.
